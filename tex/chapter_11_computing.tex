% Created: Enze Chen, June 2025
%
% Chapter 11 of the MSE 142 coursereader. This chapter introduces ideas in quantum computing as an advanced extension of the course themes.

% Uncomment the following three lines and last line to individually compile this chapter
\documentclass[12pt, english]{book}
\usepackage{142crstyle}
\begin{document}

\chapter{Quantum Computing} \label{ch:computing}
%{ \doublespacing 
As an advanced topic, we'll discuss the distinguishability (or lack thereof) of quantum particles and their topology.
While they sound abstract, these concepts are important for...
We'll just sketch the main ideas here, and encourage you to consult other resources for a more thorough treatment.


%%%%%%%%%%%%%%%%%%%%%%%%%%%%%%%%%%%%%%%%%%%%%%%%%%%%%%%%%%%%%%%%%%%%%%%%%%%%%%%%

\section{Before you begin}

This chapter builds on the following concepts, some of which we've already discussed in class, others you will likely have encountered elsewhere.
We include links to resources that may aid your review, as mastery of these concepts will allow you to get the most out of this chapter.

\begin{itemize}
	\item foo 
	\item bar 
	\item Prerequisite self-check quiz 
\end{itemize}


%%%%%%%%%%%%%%%%%%%%%%%%%%%%%%%%%%%%%%%%%%%%%%%%%%%%%%%%%%%%%%%%%%%%%%%%%%%%%%%%

\section{Qubits}



%%%%%%%%%%%%%%%%%%%%%%%%%%%%%%%%%%%%%%%%%%%%%%%%%%%%%%%%%%%%%%%%%%%%%%%%%%%%%%%%

\section{Logic gates}




%%%%%%%%%%%%%%%%%%%%%%%%%%%%%%%%%%%%%%%%%%%%%%%%%%%%%%%%%%%%%%%%%%%%%%%%%%%%%%%%

\section{Application: Superconducting qubits or spin qubits in Si}

Superconducting Qubits: Superconducting materials are among the most widely used implementations of qubits. You can explore how materials such as niobium and aluminum are employed to create Josephson junctions, essential for superconducting qubits. Discuss the quantum coherence of these materials and how their superconducting properties enable low-resistance pathways, crucial for creating stable qubit states that elevate the performance of quantum processors.

Spin Qubits in Silicon: Highlight how silicon, a well-known semiconductor, is being explored for encoding qubits using the spin of electrons or nuclei. You can cover topics on how impurities in silicon can be utilized to create qubit systems and the ongoing research to integrate quantum computing with classical silicon-based technology. Discussing the material's advantages (like the existing fabrication infrastructure from classical computing) could also resonate well with students.


%%%%%%%%%%%%%%%%%%%%%%%%%%%%%%%%%%%%%%%%%%%%%%%%%%%%%%%%%%%%%%%%%%%%%%%%%%%%%%%%

\section{Summary}
To recap, in this chapter we analyzed...

%} % for doublespacing
\end{document}